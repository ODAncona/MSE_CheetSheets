%
% Packages
%
\documentclass{article}
\usepackage[landscape]{geometry}
\usepackage{url}
\usepackage{multicol}
\usepackage{amsmath}
\usepackage{esint}
\usepackage{bigints}
\usepackage{amsfonts}
\usepackage{xcolor}
\usepackage{tikz}
\usetikzlibrary{calc}
\usetikzlibrary{decorations.pathmorphing}
\usepackage{amsmath,amssymb}
\usepackage{colortbl}
\usepackage{xcolor}
\usepackage{mathtools}
\usepackage{amsmath,amssymb}
\usepackage{enumitem}
\usepackage{xhfill}
\usepackage[french]{babel}
\usepackage[utf8]{inputenc}
\usepackage{parskip}
\usepackage[T1]{fontenc}
\usepackage{mathrsfs}
\makeatletter

%
% Math
%
\newcommand{\Real}{\mathbb R}
\newcommand{\RPlus}{\Real^{+}}
\newcommand{\norm}[1]{\left\Vert#1\right\Vert}
\newcommand{\abs}[1]{\left\vert#1\right\vert}
\newcommand{\setn}[1]{\left\{#1\right\}_{\scriptscriptstyle n \ge 1}}
\newcommand{\set}[1]{\left\{#1\right\}}
\newcommand{\seq}[1]{\left<#1\right>}
\newcommand{\eps}{\varepsilon}
\newcommand{\To}{\longrightarrow}
\newcommand{\Prob}{\rm{P}}
\newcommand{\F}{\mathcal{F}}
\newcommand{\h}{\mathcal{H}}
\newcommand{\M}{\mathcal{M}}
\newcommand{\X}{\mathcal{X}}
\newcommand{\N}{\mathcal{N}}
\newcommand{\E}{{\rm E}}
\newcommand{\Hnull}{{\rm H}_{0}}
\newcommand{\Hone}{{\rm H}_{1}}
\newcommand{\Var}{{\rm Var}}
\newcommand{\Cov}{{\rm Cov}}
\newcommand{\sign}{{\rm sign}}
\newcommand{\med}{{\rm med}}
\newcommand{\tr}{{\rm tr}}
\newcommand{\T}{{\text{\tiny \rm T}}}
\newcommand{\minf}{- \, \infty}
\newcommand{\intervalle}[4]{\mathopen{#1}#2\mathpunct{},#3\mathclose{#4}}
\newcommand{\intervalleff}[2]{\intervalle{[}{#1}{#2}{]}}
\newcommand{\intervalleof}[2]{\intervalle{]}{#1}{#2}{]}}
\newcommand{\intervallefo}[2]{\intervalle{[}{#1}{#2}{[}}
\newcommand{\intervalleoo}[2]{\intervalle{]}{#1}{#2}{[}}
\newcommand*\conj[1]{\overline{#1}}
\newcommand*\mean[1]{\overline{#1}}

%
% Setup
%
\title{APV Résumé Olivier D'Ancona}
\advance\topmargin-.8in
\advance\textheight3in
\advance\textwidth3in
\advance\oddsidemargin-1.5in
\advance\evensidemargin-1.5in
\parindent0pt
\parskip2pt

%
% Commands
%
\newcommand*\bigcdot{\mathpalette\bigcdot@{.5}}
\newcommand*\bigcdot@[2]{\mathbin{\vcenter{\hbox{\scalebox{#2}{$\m@th#1\bullet$}}}}}
\makeatother
\newcommand{\hr}{\centerline{\rule{3.5in}{1pt}}}
%\colorbox[HTML]{e4e4e4}{\makebox[\textwidth-2\fboxsep][l]{texto}
\newcommand{\nc}[2][]{%
\vspace{-.1cm}
\tikz \draw [draw=black, ultra thick, #1]
    ($(current page.center)-(0.5\linewidth,0)$) --
    ($(current page.center)+(0.5\linewidth,0)$)
    node [midway, fill=white] {#2};
}% tomado de https://tex.stackexchange.com/questions/179425/a-new-command-of-the-form-tex


%
% Styles
%
\tikzstyle{mybox} = [draw=black, fill=white, very thick,
rectangle, rounded corners, inner sep=2pt, inner ysep=7pt]
\tikzstyle{fancytitle} =[fill=black, text=white, font=\bfseries]

\newlength{\boxsize}
\setlength{\boxsize}{0.24\textwidth}

%###############################################################################################
%
%                                         Document
%
%###############################################################################################

\begin{document}

%---------------------------------
% Title
%---------------------------------
\begin{center}
    {\huge{\textbf{APV - Résumé Olivier D'Ancona}}}\\
\end{center}

\begin{multicols*}{4}
    %---------------------------------
    % Espérance
    %---------------------------------
    \begin{tikzpicture}
        \node [mybox] (box){%
            \begin{minipage}{\boxsize}
                Soit $X$ une variable aléatoire à valeurs dans $\Real$, on définit l'espérance de $X$ par: 

                Cas discret: $\mathbb{E}[X] = \displaystyle\sum_{i=1}^{\infty}{x_i \cdot p_i}$

                Cas continu: $\mathbb{E}[X] = \displaystyle\int_{-\infty}^{\infty} x \cdot f(x) \, dx $
                où $f(x)$ est la densité de probabilité de $X$.
                \nc{Propriétés}
                \begin{itemize}
                    \item $\mathbb{E}[X + Y] = \mathbb{E}[X] + \mathbb{E}[Y]$
                    \item $\mathbb{E}[aX] = a \mathbb{E}[X]$
                    \item $\mathbb{E}[X^2] = \mathbb{E}[X]^2 + \mathbb{V}[X]$
                    \item $\mathbb{E}[XY] = \mathbb{E}[X]\cdot\mathbb{E}[Y]$ if $X$ and $Y$ are independent.
                    \item $\mathbb{E}[c] = c$ \text{ if $c$ is a constant.}
                \end{itemize}
            \end{minipage}
        };
        \node[fancytitle, right=10pt] at (box.north west) {Espérance};
    \end{tikzpicture}
    %---------------------------------
    
    %---------------------------------
    % Variance
    %---------------------------------
    \begin{tikzpicture}
        \node [mybox] (box){%
            \begin{minipage}{\boxsize}
                Soit $X$ une variable aléatoire à valeurs dans $\Real$.Cas discret:
                \vspace{-.2cm}
                \begin{align*}
                    \mathbb{V}[X] &= \sum_{i=1}^{\infty}{(x_i-\mathbb{E}[X])^2 \cdot p_i} \\
                    &= \mathbb{E}[(X_i-\mu)^2]= \dfrac{1}{n} \sum_{i=1}^{n}(X_i-\bar{X})^2
                \end{align*}
                Cas continu:
                Cas discret:\\
                $$\mathbb{V}[X] = \displaystyle\int_{-\infty}^{\infty} (x-\mathbb{E}[X])^2 \cdot f(x) \, dx $$
                Lien entre écart-type et variance: $\mathbb{V}[X] = \sigma^2$\\
                \nc{Propriétés}
                \begin{itemize}
                    \item $\mathbb{V}[X] = \mathbb{E}[(X - \mathbb{E}[X])^2] = \mathbb{E}[X^2] - \mathbb{E}[X]^2$
                    \item $\mathbb{V}[X + Y] = \mathbb{V}[X] + \mathbb{V}[Y]$ \text{ if $X$ and $Y$ are independent.}
                    \item $\mathbb{V}[aX] = a^2 \mathbb{V}[X]$
                    \item $\mathbb{V}[XY] = \mathbb{V}[X]\cdot\mathbb{V}[Y] + \mathbb{V}[X]\cdot\mathbb{E}[Y]^2 + \mathbb{V}[Y]\cdot\mathbb{E}[X]^2$ \text{ if $X$ and $Y$ are independent.}
                    \item $\mathbb{V}[c] = 0$ \text{ if $c$ is a constant.}
                \end{itemize}
            \end{minipage}
        };
        \node[fancytitle, right=10pt] at (box.north west) {Variance};
    \end{tikzpicture}
    %---------------------------------

    %---------------------------------
    % Loi de probabilité
    %---------------------------------
    \begin{tikzpicture}
        \node [mybox] (box){%
        \begin{tabular}[width=0.4\boxsize]{|c|c|c|c|c|}
            \hline
            Loi & Param & $\mathbb{E}$ & $\mathbb{V}$ & Support \\
            Bernoulli & $p$ & $p$ & $p(1-p)$ & ${0,1}$ \\
            Binomiale & $n,p$ & $np$ & $np(1-p)$ & ${0,1}$ \\
            Uniforme & $a,b$ & $\frac{a+b}{2}$ & $\frac{(b-a)^2}{12}$ & $[a,b]$ \\
            Normale & $\mu,\sigma$ & $\mu$ & $\sigma^2$ & $R$ \\
            Exponentielle & $\lambda$ & $\frac{1}{\lambda}$ & $\frac{1}{\lambda^2}$ & $[0,\infty)$ \\
            Poisson & $\lambda$ & $\lambda$ & $\lambda$ & $[0,\infty)$ \\
            Géométrique & $p$ & $\frac{1}{p}$ & $\frac{1}{p^2}$ & $[0,\infty)$ \\
            \hline
        \end{tabular}
            \begin{minipage}{\boxsize}
            \end{minipage}
        };
        \node[fancytitle, right=10pt] at (box.north west) {Loi de probabilité};
    \end{tikzpicture}
    %---------------------------------
    
    %---------------------------------
    % Likelihood
    %---------------------------------
    \begin{tikzpicture}
        \node [mybox] (box){%
            \begin{minipage}{\boxsize}
                Soit $X$ une variable aléatoire à valeurs dans $\mathbb{R}$, on définit la likelihood de $\theta$ par:
                
                $$L(\theta) = \prod_{i=1}^n f(x_i | \theta)$$
                
                où $f(x)$ est la densité de probabilité de $X$.
                
                Souvent, on utilise le logarithme de la likelihood:\\
                
                $$\log L(\theta) = \sum_{i=1}^n \log f(x_i | \theta)$$
            \end{minipage}
        };
        \node[fancytitle, right=10pt] at (box.north west) {Likelihood};
    \end{tikzpicture}
    %---------------------------------

    %---------------------------------
    % MLE
    %---------------------------------
    \begin{tikzpicture}
        \node [mybox] (box){%
            \begin{minipage}{\boxsize}
                $\hat{\theta}_{MLE}$ est de $\theta$ est le paramètre qui maximise la likelihood de $\theta$.
                
                Soit $\theta$ un paramètre d'une loi de probabilité, on définit le MLE de $\theta$ par:
                
                $$\hat{\theta} = \arg\max_{\theta} L(\theta)$$
                
                où $L(\theta)$ est la likelihood de $\theta$.
                
                \nc{Procédure}
                1. Définir la likelihood de $\theta$ $L(\theta)$.
    
                2. On passe au logarithme de la likelihood. $l(\theta) = \log L(\theta)$

                3. On dérive la log-likelihood par rapport à $\theta$

                4. On cherche $\hat{\theta}$ tel que $\frac{\partial}{\partial \theta} l(\hat{\theta}) = 0$.

                5. On vérifie que $\frac{\partial^2}{\partial \theta^2}l(\hat{\theta}) < 0$

            \end{minipage}
        };
        \node[fancytitle, right=10pt] at (box.north west) {MLE};
    \end{tikzpicture}
    %---------------------------------

    %---------------------------------
    % Régression Linéaire
    %---------------------------------
    \begin{tikzpicture}
        \node [mybox] (box){%
            \begin{minipage}{\boxsize}
                Soit un tableau de données: 

                $x$ = Soap(g) , $y$ = Height(cm) , $x \cdot y$ , $x^2$	  

                $$ X = [1, Soap] $$
                $$ X^TX = \begin{bmatrix} n & \sum{x_i} \\ \sum{x_i} & \sum{x_i^2} \end{bmatrix} = \begin{bmatrix} 7 & 38.5 \\ 38.5 & 218.95 \end{bmatrix} $$
                $$ X^Ty = \begin{bmatrix} \sum{y_i} \\ \sum{x_iy_i} \end{bmatrix} = \begin{bmatrix} 348 \\ 1975 \end{bmatrix} $$
                $$ \hat{\theta} = (X^TX)^{-1}X^Ty = \begin{bmatrix} -2.67 \\ 9.51 \end{bmatrix} $$
                Inverse d'une matrice 2x2 :

                $$ \begin{bmatrix} a & b \\ c & d \end{bmatrix}^{-1} = \dfrac{1}{ad-bc} \begin{bmatrix} d & -b \\ -c & a \end{bmatrix} $$
            \end{minipage}
        };
        \node[fancytitle, right=10pt] at (box.north west) {Régression Linéaire};
    \end{tikzpicture}
    %---------------------------------

    %---------------------------------
    % Normale
    %---------------------------------
    \begin{tikzpicture}
        \node [mybox] (box){%
            \begin{minipage}{\boxsize}
                \begin{enumerate}
                    \item $P(X \leq x) = \Phi(x) = \displaystyle\int_{-\infty}^{x} N(0,1)dx$
                    \item $\Phi(x) = 1 - \Phi(-x)$
                    \item $F_x(x) = \Phi\left(\dfrac{x-\mu}{\sigma}\right)$ est une loi normale centrée réduite
                \end{enumerate}
            \end{minipage}
        };
        \node[fancytitle, right=10pt] at (box.north west) {Normale};
    \end{tikzpicture}
    %---------------------------------

%###############################################################################################
%                                         Deuxième Page
%###############################################################################################


\end{multicols*}
    
\end{document}
