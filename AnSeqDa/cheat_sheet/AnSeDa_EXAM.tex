%
% Packages
%
\documentclass{article}
\usepackage[landscape]{geometry}
\usepackage{url}
\usepackage{multicol}
\usepackage{amsmath}
\usepackage{esint}
\usepackage{bigints}
\usepackage{amsfonts}
\usepackage{xcolor}
\usepackage{tikz}
\usetikzlibrary{calc}
\usetikzlibrary{decorations.pathmorphing}
\usepackage{amsmath,amssymb}
\usepackage{colortbl}
\usepackage{xcolor}
\usepackage{mathtools}
\usepackage{amsmath,amssymb}
\usepackage{enumitem}
\usepackage{xhfill}
\usepackage[french]{babel}
\usepackage[utf8]{inputenc}
\usepackage{parskip}
\usepackage[T1]{fontenc}
\usepackage{mathrsfs}
\makeatletter

%
% Math
%
\newcommand{\Real}{\mathbb R}
\newcommand{\RPlus}{\Real^{+}}
\newcommand{\norm}[1]{\left\Vert#1\right\Vert}
\newcommand{\abs}[1]{\left\vert#1\right\vert}
\newcommand{\setn}[1]{\left\{#1\right\}_{\scriptscriptstyle n \ge 1}}
\newcommand{\set}[1]{\left\{#1\right\}}
\newcommand{\seq}[1]{\left<#1\right>}
\newcommand{\eps}{\varepsilon}
\newcommand{\To}{\longrightarrow}
\newcommand{\Prob}{\rm{P}}
\newcommand{\F}{\mathcal{F}}
\newcommand{\h}{\mathcal{H}}
\newcommand{\M}{\mathcal{M}}
\newcommand{\X}{\mathcal{X}}
\newcommand{\N}{\mathcal{N}}
\newcommand{\E}{\mathrm{E}}
\newcommand{\Hnull}{{\rm H}_{0}}
\newcommand{\Hone}{{\rm H}_{1}}
\newcommand{\Var}{\mathrm{Var}}
\newcommand{\Cov}{\mathrm{Cov}}
\newcommand{\sign}{{\rm sign}}
\newcommand{\med}{{\rm med}}
\newcommand{\tr}{{\rm tr}}
\newcommand{\T}{{\text{\tiny \rm T}}}
\newcommand{\minf}{- \, \infty}
\newcommand{\intervalle}[4]{\mathopen{#1}#2\mathpunct{},#3\mathclose{#4}}
\newcommand{\intervalleff}[2]{\intervalle{[}{#1}{#2}{]}}
\newcommand{\intervalleof}[2]{\intervalle{]}{#1}{#2}{]}}
\newcommand{\intervallefo}[2]{\intervalle{[}{#1}{#2}{[}}
\newcommand{\intervalleoo}[2]{\intervalle{]}{#1}{#2}{[}}
\newcommand*\conj[1]{\overline{#1}}
\newcommand*\mean[1]{\overline{#1}}

%
% Setup
%
\title{AnSeDa Résumé Olivier D'Ancona}
\advance\topmargin-.8in
\advance\textheight3in
\advance\textwidth3in
\advance\oddsidemargin-1.5in
\advance\evensidemargin-1.5in
\parindent0pt
\parskip2pt

%
% Commands
%
\newcommand*\bigcdot{\mathpalette\bigcdot@{.5}}
\newcommand*\bigcdot@[2]{\mathbin{\vcenter{\hbox{\scalebox{#2}{$\m@th#1\bullet$}}}}}
\makeatother
\newcommand{\hr}{\centerline{\rule{3.5in}{1pt}}}
%\colorbox[HTML]{e4e4e4}{\makebox[\textwidth-2\fboxsep][l]{texto}
\newcommand{\nc}[2][]{%
\vspace{-.1cm}
\tikz \draw [draw=black, ultra thick, #1]
    ($(current page.center)-(0.5\linewidth,0)$) --
    ($(current page.center)+(0.5\linewidth,0)$)
    node [midway, fill=white] {#2};
}% tomado de https://tex.stackexchange.com/questions/179425/a-new-command-of-the-form-tex


%
% Styles
%
\tikzstyle{mybox} = [draw=black, fill=white, very thick,
rectangle, rounded corners, inner sep=2pt, inner ysep=7pt]
\tikzstyle{fancytitle} =[fill=black, text=white, font=\bfseries]

\newlength{\boxsize}
\setlength{\boxsize}{0.24\textwidth}

%###############################################################################################
%
%                                         Document
%
%###############################################################################################

\begin{document}

%---------------------------------
% Title
%---------------------------------
\begin{center}
    {\huge{\textbf{AnSeDa - Résumé Olivier D'Ancona}}}\\
\end{center}

\begin{multicols*}{4}
    %---------------------------------
    % Espérance
    %---------------------------------
    \begin{tikzpicture}
        \node [mybox] (box){%
            \begin{minipage}{\boxsize}
                Soit $X$ une variable aléatoire à valeurs dans $\Real$, on définit l'espérance de $X$ par: 
                
                Cas discret: $ \E[X] = \displaystyle\sum_{i=1}^{\infty}{x_i \cdot p_i}$
                
                Cas continu: $\E[X] = \displaystyle\int_{-\infty}^{\infty} x \cdot f(x) \, dx $
                où $f(x)$ est la densité de probabilité de $X$.
                \nc{Propriétés}
                \begin{itemize}
                    \item $ \E[X + Y] =  \E[X] +  \E[Y]$
                    \item $ \E[aX] = a  \E[X]$
                    \item $ [X^2] =  [X]^2 + \Var[X]$
                    \item $ \E[XY] =  \E[X]\cdot \E[Y]$ if $X$ and $Y$ are independent.
                    \item $ \E[c] = c$ \text{ if $c$ is a constant.}
                \end{itemize}
            \end{minipage}
        };
        \node[fancytitle, right=10pt] at (box.north west) {Espérance};
    \end{tikzpicture}
    %---------------------------------
    
    %---------------------------------
    % Variance
    %---------------------------------
    \begin{tikzpicture}
        \node [mybox] (box){%
            \begin{minipage}{\boxsize}
                Soit $X$ une variable aléatoire à valeurs dans $\Real$. Cas discret // continu:
                \vspace{-.2cm}
                \begin{align*}
                    \Var[X] & = \sum_{i=1}^{\infty}{(x_i- \E[X])^2 \cdot p_i}                        \\
                            & =  \E[(X_i-\mu)^2]= \dfrac{1}{n} \sum_{i=1}^{n}(X_i-\bar{X})^2         \\
                    \Var[X] & = \displaystyle\int_{-\infty}^{\infty} (x- \E[X])^2 \cdot f(x) \, dx 
                \end{align*}
                $\Var[X] = \sigma^2$\\
                \nc{Propriétés}
                \begin{itemize}
                    \item $\Var[X] =  \E[(X -  \E[X])^2] =  \E[X^2] -  \E[X]^2$
                    \item $\Var[aX] = a^2 \Var[X]$
                    \item $\Var[XY] = \Var[X]\cdot\Var[Y] + \Var[X]\cdot \E[Y]^2 + \Var[Y]\cdot \E[X]^2$ \text{ if $X$ and $Y$ are independent.}
                    \item $\Var[X + Y] = \Var[X] + \Var[Y]$ + $2\Cov[X, Y]$
                    \item $\Var[c] = 0$ \text{ if $c$ is a constant.}
                \end{itemize}
            \end{minipage}
        };
        \node[fancytitle, right=10pt] at (box.north west) {Variance};
    \end{tikzpicture}
    %---------------------------------
    
    
    %---------------------------------
    % Covariance
    %---------------------------------
    \begin{tikzpicture}
        \node [mybox] (box){%
            \begin{minipage}{\boxsize}
                Soit $X$ et $Y$ deux variables aléatoires, la covariance entre $X$ et $Y$ est définie par:
                \begin{align*}
                    \Cov(X, Y) & =  \E[(X -\E[X])(Y -\E[Y])] \\
                               & =  \E[XY] -\E[X] \E[Y]
                \end{align*}
                
                \nc{Propriétés}
                \begin{itemize}
                    \item $\Cov(X, c) = 0$ si $c$ est une constante
                    \item $\Cov(X, Y) = \Cov(Y, X)$
                    \item $\Cov(X, X) = \text{Var}(X)$
                    \item $\Cov(aX + b, cY + d) = ac \cdot \Cov(X, Y)$
                    \item $\Cov(X+Y,Z) = \Cov(X, Z) + \Cov(Y, Z)$
                    \item $\Cov(X, Y) = 0$ Si $X$ et $Y$ sont indépendants
                \end{itemize}
            \end{minipage}
        };
        \node[fancytitle, right=10pt] at (box.north west) {Covariance};
    \end{tikzpicture}
    %---------------------------------
    
    
    %---------------------------------
    % Correlation
    %---------------------------------
    \begin{tikzpicture}
        \node [mybox] (box){%
            \begin{minipage}{\boxsize}
                Soit $X$ et $Y$ deux variables aléatoires, le coefficient de corrélation entre $X$ et $Y$ est défini par:
                \begin{align*}
                    \rho_{XY} & = \frac{\Cov(X, Y)}{\sqrt{\Var(X) \Var(Y)}}         = \frac{\Cov(X, Y)}{\sigma_X \sigma_Y}
                \end{align*}
                
                \nc{Propriétés}
                \begin{itemize}
                    \item $\rho_{XY} = \rho_{YX}$
                    \item $-1 \leq \rho_{XY} \leq 1$
                    \item $\rho_{XY} = 1$ ou $-1$ indique une corrélation linéaire parfaite
                    \item $\rho_{XY} = 0$ si $X$ et $Y$ sont indépendants linéairement
                    \item Invariance des changements linéaires : $\rho_{aX+b, cY+d} = \rho_{XY}$ pour tout $a, b, c, d$ réels avec $ac \neq 0$
                \end{itemize}
            \end{minipage}
        };
        \node[fancytitle, right=10pt] at (box.north west) {Corrélation};
    \end{tikzpicture}
    %---------------------------------
    
    %---------------------------------
    % Stationnarité
    %---------------------------------
    \begin{tikzpicture}
        \node [mybox] (box){%
            \begin{minipage}{\boxsize}
                La stationnarité d'un processus aléatoire décrit comment ses propriétés statistiques restent constantes au fil du temps. Pour qu'une série temporelle soit stationnaire, elle doit présenter quatre propriétés constantes dans le temps :
                \begin{enumerate}
                    \item Moyenne constante
                    \item Variance constante
                    \item Structure d'autocorrélation constante
                    \item Aucun composant périodique (saisonnalité)
                \end{enumerate}
                L'autocorrélation signifie que la mesure actuelle de la série temporelle est corrélée à une mesure passée.
                
                \nc{Stationnarité au sens strict} Un processus $\{X_i\}$ est strictement stationnaire si :
                \begin{itemize}
                    \item La distribution de $X_i$ est la même que celle de $X_j$ pour $i \neq j$.
                    \item Les implications sont que $\E[X_i]$ et $\Var(X_i)$ sont constants et ne dépendent pas de $i$.
                    \item La distribution conjointe de $(X_i, X_j)$ est la même que celle de $(X_{i+k}, X_{j+k})$.
                    \item L'autocovariance $\Cov(X_i, X_{i+j})$ ne dépend que de $j$.
                \end{itemize}
                
                \nc{Stationnarité au sens large} Un processus $\{X_i\}$ est largement stationnaire si :
                \begin{itemize}
                    \item $\E[X_i]$ et $\Var(X_i)$ sont constants et ne dépendent pas de $i$.
                    \item L'autocovariance $\Cov(X_i, X_{i+j}) = \E[X_i X_{i+j}] - \E[X_i]\E[X_{i+j}]$ ne dépend que de $j$.
                \end{itemize}
            \end{minipage}
        };
        \node[fancytitle, right=10pt] at (box.north west) {Stationnarité};
    \end{tikzpicture}
    %---------------------------------
    
    
    
    %---------------------------------
    % Loi de probabilité
    %---------------------------------
    \begin{tikzpicture}
        \node [mybox] (box){%
            \resizebox{\boxsize}{!}{%
                \begin{tabular}{|c|c|c|c|c|}
                    \hline
                    Loi           & Param        & $ \E[X]$            & $\Var(X)$             & Support               \\
                    Bernoulli     & $p$          & $p$                 & $p(1-p)$              & $\{0,1\}$             \\
                    Binomiale     & $n,p$        & $np$                & $np(1-p)$             & $\{0, \ldots, n\}$    \\
                    Uniforme      & $a,b$        & $\frac{a+b}{2}$     & $\frac{(b-a)^2}{12}$  & $[a,b]$               \\
                    Normale       & $\mu,\sigma$ & $\mu$               & $\sigma^2$            & $\mathbb{R}$          \\
                    Exponentielle & $\lambda$    & $\frac{1}{\lambda}$ & $\frac{1}{\lambda^2}$ & $[0,\infty)$          \\
                    Poisson       & $\lambda$    & $\lambda$           & $\lambda$             & $\{0, 1, 2, \ldots\}$ \\
                    Géométrique   & $p$          & $\frac{1}{p}$       & $\frac{1-p}{p^2}$     & $\{0, 1, 2, \ldots\}$ \\
                    \hline
                \end{tabular}
            }
            \begin{minipage}{\boxsize}
            \end{minipage}
        };
        \node[fancytitle, right=10pt] at (box.north west) {Loi de probabilité};
    \end{tikzpicture}
    %---------------------------------
    
    
    %---------------------------------
    % Likelihood
    %---------------------------------
    \begin{tikzpicture}
        \node [mybox] (box){%
            \begin{minipage}{\boxsize}
                Soit $X$ une variable aléatoire à valeurs dans $\mathbb{R}$, on définit la likelihood de $\theta$ par:
                
                $$L(\theta) = \prod_{i=1}^n f(x_i | \theta)$$
                
                où $f(x)$ est la densité de probabilité de $X$.
                
                Souvent, on utilise le logarithme de la likelihood:\\
                
                $$\log L(\theta) = \sum_{i=1}^n \log f(x_i | \theta)$$
            \end{minipage}
        };
        \node[fancytitle, right=10pt] at (box.north west) {Likelihood};
    \end{tikzpicture}
    %---------------------------------
    
    %---------------------------------
    % MLE
    %---------------------------------
    \begin{tikzpicture}
        \node [mybox] (box){%
            \begin{minipage}{\boxsize}
                $\hat{\theta}_{MLE}$ est de $\theta$ est le paramètre qui maximise la likelihood de $\theta$.
                
                Soit $\theta$ un paramètre d'une loi de probabilité, on définit le MLE de $\theta$ par:
                
                $$\hat{\theta} = \arg\max_{\theta} L(\theta)$$
                
                où $L(\theta)$ est la likelihood de $\theta$.
                
                \nc{Procédure}
                1. Définir la likelihood de $\theta$ $L(\theta)$.
                
                2. On passe au logarithme de la likelihood. $l(\theta) = \log L(\theta)$
                
                3. On dérive la log-likelihood par rapport à $\theta$
                
                4. On cherche $\hat{\theta}$ tel que $\frac{\partial}{\partial \theta} l(\hat{\theta}) = 0$.
                
                5. On vérifie que $\frac{\partial^2}{\partial \theta^2}l(\hat{\theta}) < 0$
                
            \end{minipage}
        };
        \node[fancytitle, right=10pt] at (box.north west) {MLE};
    \end{tikzpicture}
    %---------------------------------
    
    %---------------------------------
    % Régression Linéaire
    %---------------------------------
    \begin{tikzpicture}
        \node [mybox] (box){%
            \begin{minipage}{\boxsize}
                Soit un tableau de données: 
                
                $x$ = Soap(g) , $y$ = Height(cm) , $x \cdot y$ , $x^2$
                
                $$ X = [1, Soap] $$
                $$ X^TX = \begin{bmatrix} n & \sum{x_i} \\ \sum{x_i} & \sum{x_i^2} \end{bmatrix} = \begin{bmatrix} 7 & 38.5 \\ 38.5 & 218.95 \end{bmatrix} $$
                $$ X^Ty = \begin{bmatrix} \sum{y_i} \\ \sum{x_iy_i} \end{bmatrix} = \begin{bmatrix} 348 \\ 1975 \end{bmatrix} $$
                $$ \hat{\theta} = (X^TX)^{-1}X^Ty = \begin{bmatrix} -2.67 \\ 9.51 \end{bmatrix} $$
                Inverse d'une matrice 2x2 :
                
                $$ \begin{bmatrix} a & b \\ c & d \end{bmatrix}^{-1} = \dfrac{1}{ad-bc} \begin{bmatrix} d & -b \\ -c & a \end{bmatrix} $$
            \end{minipage}
        };
        \node[fancytitle, right=10pt] at (box.north west) {Régression Linéaire};
    \end{tikzpicture}
    %---------------------------------
    
    %---------------------------------
    % Normale
    %---------------------------------
    \begin{tikzpicture}
        \node [mybox] (box){%
            \begin{minipage}{\boxsize}
                \begin{enumerate}
                    \item $P(X \leq x) = \Phi(x) = \displaystyle\int_{-\infty}^{x} N(0,1)dx$
                    \item $\Phi(x) = 1 - \Phi(-x)$
                    \item $F_x(x) = \Phi\left(\dfrac{x-\mu}{\sigma}\right)$ est une loi normale centrée réduite
                \end{enumerate}
            \end{minipage}
        };
        \node[fancytitle, right=10pt] at (box.north west) {Normale};
    \end{tikzpicture}
    %---------------------------------
    
    %---------------------------------
    % TOPIC
    %---------------------------------
    \begin{tikzpicture}
        \node [mybox] (box){%
            \begin{minipage}{\boxsize}
            \end{minipage}
        };
        \node[fancytitle, right=10pt] at (box.north west) {TOPIC};
    \end{tikzpicture}
    %---------------------------------
    %###############################################################################################
    %                                         Deuxième Page
    %###############################################################################################
    
    
\end{multicols*}

\end{document}
