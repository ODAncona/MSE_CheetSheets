%
% Packages
%
\documentclass[10pt]{article} 
\usepackage[landscape]{geometry}
\usepackage{url}
\usepackage{multicol}
\usepackage{amsmath}
\usepackage{esint}
\usepackage{bigints}
\usepackage{amsfonts}
\usepackage{xcolor}
\usepackage{tikz}
\usepackage{amsmath,amssymb}
\usepackage{colortbl}
\usepackage{xcolor}
\usepackage{mathtools}
\usepackage{amsmath,amssymb}
\usepackage{enumitem}
\usepackage{xhfill}
\usepackage[french]{babel}
\usepackage[utf8]{inputenc}
\usepackage{parskip}
\usepackage[T1]{fontenc}
\usepackage{mathrsfs}
\usepackage{pgfplots}
\usepackage{algorithm} % For writing pseudocode
\usepackage{algpseudocode} % For pseudocode layout
\usetikzlibrary{calc}
\usetikzlibrary{decorations.pathmorphing}
\usetikzlibrary{pgfplots.groupplots}
\pgfplotsset{compat=1.17}
\makeatletter

%
% Math
%
\newcommand{\Real}{\mathbb R}
\newcommand{\RPlus}{\Real^{+}}
\newcommand{\norm}[1]{\left\Vert#1\right\Vert}
\newcommand{\abs}[1]{\left\vert#1\right\vert}
\newcommand{\setn}[1]{\left\{#1\right\}_{\scriptscriptstyle n \ge 1}}
\newcommand{\set}[1]{\left\{#1\right\}}
\newcommand{\seq}[1]{\left<#1\right>}
\newcommand{\eps}{\varepsilon}
\newcommand{\To}{\longrightarrow}
\newcommand{\Prob}{\rm{P}}
\newcommand{\F}{\mathcal{F}}
\newcommand{\h}{\mathcal{H}}
\newcommand{\M}{\mathcal{M}}
\newcommand{\X}{\mathcal{X}}
\newcommand{\N}{\mathcal{N}}
\newcommand{\E}{{\rm E}}
\newcommand{\Hnull}{{\rm H}_{0}}
\newcommand{\Hone}{{\rm H}_{1}}
\newcommand{\Var}{{\rm Var}}
\newcommand{\Cov}{{\rm Cov}}
\newcommand{\sign}{{\rm sign}}
\newcommand{\med}{{\rm med}}
\newcommand{\tr}{{\rm tr}}
\newcommand{\T}{{\text{\tiny \rm T}}}
\newcommand{\minf}{- \, \infty}
\newcommand{\intervalle}[4]{\mathopen{#1}#2\mathpunct{},#3\mathclose{#4}}
\newcommand{\intervalleff}[2]{\intervalle{[}{#1}{#2}{]}}
\newcommand{\intervalleof}[2]{\intervalle{]}{#1}{#2}{]}}
\newcommand{\intervallefo}[2]{\intervalle{[}{#1}{#2}{[}}
\newcommand{\intervalleoo}[2]{\intervalle{]}{#1}{#2}{[}}
\newcommand*\conj[1]{\overline{#1}}
\newcommand*\mean[1]{\overline{#1}}

%
% Setup
%
\geometry{
  top=5mm,    % Minimal top margin
  %bottom=5mm, % Minimal bottom margin
  left=5mm,   % Minimal left margin
  right=5mm,   % Minimal right margin
  textheight=220mm
}
\title{Deep Learning Cheat Sheet}

% Paragraph and column formatting
\setlength{\parindent}{0pt}          % No indentation at the start of a paragraph
\setlength{\parskip}{1pt plus 1pt minus 0.5pt}  % Moderate space between paragraphs
\setlength{\columnsep}{10pt}         % Moderate space between columns

% Line and baseline spacing
\setlength{\baselineskip}{2pt}      % Adequate baseline skip
\linespread{0.4}                     % Reset line spacing to normal

% Space around displayed math
\setlength{\abovedisplayskip}{3pt}
\setlength{\belowdisplayskip}{3pt}
\setlength{\abovedisplayshortskip}{3pt}
\setlength{\belowdisplayshortskip}{3pt}

%
% Commands
%
\newcommand*\bigcdot{\mathpalette\bigcdot@{.5}}
\newcommand*\bigcdot@[2]{\mathbin{\vcenter{\hbox{\scalebox{#2}{$\m@th#1\bullet$}}}}}
\makeatother
\newcommand{\hr}{\centerline{\rule{3.5in}{1pt}}}
%\colorbox[HTML]{e4e4e4}{\makebox[\textwidth-2\fboxsep][l]{texto}
\newcommand{\nc}[2][]{%
\vspace{-.16cm}
\tikz \draw [draw=black, ultra thick, #1]
    ($(current page.center)-(0.5\linewidth,0)$) --
    ($(current page.center)+(0.5\linewidth,0)$)
    node [midway, fill=white] {#2};
}% tomado de https://tex.stackexchange.com/questions/179425/a-new-command-of-the-form-tex


%
% Styles
%
\tikzstyle{mybox} = [draw=black, fill=white, very thick,
rectangle, rounded corners, inner sep=2pt, inner ysep=7pt]
\tikzstyle{fancytitle} =[fill=black, text=white, font=\bfseries]

\newlength{\boxsize}
\setlength{\boxsize}{0.247\textwidth}
\raggedcolumns
%###############################################################################################
%
%                                         Document
%
%###############################################################################################



\begin{document}

%---------------------------------
% Title
%---------------------------------
\begin{center}
    {\huge{\textbf{Deep Learning Cheat Sheet}}}\\
\end{center}
\vspace{-0.1cm}

\begin{multicols*}{4}
    
    %---------------------------------
    % Evaluation Metrics
    %---------------------------------
    \begin{tikzpicture}
        \node [mybox] (box){%
            \begin{minipage}{0.247\textwidth}
                $Accuracy = \dfrac{TP + TN}{ TP + TN + FP + FN}$\\
                $Precision = \dfrac{TP}{ TP + FP}$\\
                $Recall = \dfrac{TP}{ TP + FN}$ \\
                $Specificity = \dfrac{TP}{TN + FP}$\\
                $Fscore = \dfrac{2\cdot Precision \cdot Recall}{ Precision + Recall}$\\
                $macro$ $average = \dfrac{1}{n} \sum_{i=1}^{n} avg_i$\\
            \end{minipage}
        };
        \node[fancytitle, right=10pt] at (box.north west) {Evaluation Metrics};
    \end{tikzpicture}
    %---------------------------------
    
    %---------------------------------
    % Activation Functions
    %---------------------------------
    \begin{tikzpicture}
        \node [mybox] (box){%
            \begin{minipage}{0.247\textwidth}
                \textbf{Sigmoid} : $f(z) = \frac{1}{1 + e^{-z}}$ --- Smooth and differentiable. Used in output layers for binary classification.\\
                \textbf{Hyperbolic Tangent (tanh)}
                : $f(z) = \tanh(z)$ --- Smooth, differentiable, output centered around 0. Used in LSTM.\\
                
                \textbf{Rectified Linear Unit (ReLU)}
                : $f(z) = \max(0, z)$ --- Non-linear, used as a standard, but has dying units problem for $z<0$.\\
                
                \textbf{Leaky ReLU}
                : $f(z) = \begin{cases} z & \text{if } z \geq 0 \\ \alpha z & \text{if } z < 0 \end{cases}$ --- Addresses dying units problem with a small $\alpha$ (typical $\alpha = 0.01$).\\
                
                
                \textbf{Exponential Linear Unit (ELU)}
                : $f(z) = \begin{cases} z & \text{if } z \geq 0 \\ \alpha(e^z - 1) & \text{if } z < 0 \end{cases}$ --- Similar to Leaky ReLU but more computationally expensive.\\
                
                
                \textbf{Softmax}
                : $f(z_i) = \frac{e^{z_i}}{\sum_{j=0}^{K-1} e^{z_j}}$ --- Used in the last layer for multi-class classification, outputs a probability distribution.
            \end{minipage}
        };
        \node[fancytitle, right=10pt] at (box.north west) {Activation Functions};
    \end{tikzpicture}
    %---------------------------------
    
    %---------------------------------
    % Data Preparation
    %---------------------------------
    \begin{tikzpicture}
        \node [mybox] (box){%
            \begin{minipage}{0.247\textwidth}
                \textbf{Min-max [0,1]}: $x' = \dfrac{\left(x-x_{min}\right)}{\left(x_{max}-x_{min}\right)}$\\
                \textbf{Min-max [-1,1]}: $x' = 2 \cdot min\_max(x) -1$ \\
                min-max doesn't handle outliers.\\
                \textbf{Z-norm}: $x' = \dfrac{\left(x-\mu\right)}{\sigma}$\\
                \textbf{Scaling \& Centering}\\
                Scaling improves the numerical stability, the convergence speed and accuracy of the learning algorithms. Centering improves the robustness of the learning algorithms
            \end{minipage}
        };
        \node[fancytitle, right=10pt] at (box.north west) {Data Preparation};
    \end{tikzpicture}
    %---------------------------------
    
    
    
    %---------------------------------
    % Gradient Descent
    %---------------------------------
    \begin{tikzpicture}
        \node [mybox] (box){%
            \begin{minipage}{0.247\textwidth}
                \begin{algorithmic}[1]
                    \State Initialize parameter vector $\theta_0$
                    \Repeat
                    \State Compute the gradient of the cost function at current position $\theta_t$: $\nabla_{\theta} J(\theta_t)$
                    \State Update the parameter vector by moving against the gradient:
                    $
                        \theta_{t+1} = \theta_t - \alpha \cdot \nabla_{\theta} J(\theta_t)
                    $
                    \State where $\alpha$ is the learning rate.
                    \Until{change in $\theta$ is small}
                \end{algorithmic}
                \nc{MSE}\\
                $$
                    J_{MSE}(\theta) = \frac{1}{2m} \sum_{i=1}^m (\hat{y}(i) - y(i))^2
                $$
                where:
                \begin{itemize}
                    \item \( \hat{y}(i) = h_{\theta}(x(i)) \) is the prediction of the model,
                    \item \( y(i) \) is the true outcome,
                    \item \( m \) is the number of training examples.
                \end{itemize}
                \begin{multline*}
                    \\\nabla_w J_{MSE}(w, b) =\\ \frac{1}{m} \sum_{i=1}^m \hat{y}(i) \cdot (1 - \hat{y}(i)) \cdot (\hat{y}(i) - y(i)) \cdot x(i)\\
                    \nabla_b J_{MSE}(w, b) = \\\frac{1}{m} \sum_{i=1}^m \hat{y}(i) \cdot (1 - \hat{y}(i)) \cdot (\hat{y}(i) - y(i))
                \end{multline*}
                
                \nc{Cross Entropy}
                \begin{multline*}J_{CE}(\theta) = - \sum_{i=1}^m y(i) \cdot \log h_\theta(x(i)) + \\(1 - y(i)) \cdot \log (1 - h_\theta(x(i))) \end{multline*}
                where:
                \begin{itemize}
                    \item \( p_\theta(y(i) \mid x(i)) \) is the probability model parameterized by \( \theta \), predicting the probability of the true class \( y(i) \) given the input \( x(i) \),
                    \item \( m \) is the number of observations or data points in the dataset.
                \end{itemize}
                $\nabla_w J_{CE}(w, b) = \frac{1}{m} \sum_{i=1}^m (\hat{y}(i) - y(i)) \cdot x(i)$
                
                $\nabla_b J_{CE}(w, b) = \frac{1}{m} \sum_{i=1}^m (\hat{y}(i) - y(i))$
            \end{minipage}
        };
        \node[fancytitle, right=10pt] at (box.north west) {Gradient Descent};
    \end{tikzpicture}
    %---------------------------------
    
    %---------------------------------
    % Performance Measures
    %---------------------------------
    \begin{tikzpicture}
        \node [mybox] (box){%
            \begin{minipage}{0.247\textwidth}
                \textbf{Matrice de Confusion}\\
                \textbf{Confusion Table}\\
                \textbf{ROC}\\
                \textbf{Precision}\\
                \textbf{Recall}\\
            \end{minipage}
        };
        \node[fancytitle, right=10pt] at (box.north west) {Performance Measures};
    \end{tikzpicture}
    %---------------------------------
    
    %---------------------------------
    % Bias & Variance
    %---------------------------------
    \begin{tikzpicture}
        \node [mybox] (box){%
            \begin{minipage}{0.247\textwidth}
                \textbf{Model Selection}\\
                \textbf{Bias}\\
                \textbf{Variance}\\
            \end{minipage}
        };
        \node[fancytitle, right=10pt] at (box.north west) {Bias \& Variance};
    \end{tikzpicture}
    %---------------------------------
    
    %---------------------------------
    % Theory
    %---------------------------------
    \begin{tikzpicture}
        \node [mybox] (box){%
            \begin{minipage}{0.247\textwidth}
                \textbf{Compute Graph}\\
                \textbf{Universal Approximation Theorem}\\
            \end{minipage}
        };
        \node[fancytitle, right=10pt] at (box.north west) {Theory};
    \end{tikzpicture}
    %---------------------------------
    
    %---------------------------------
    % Curse of Dimensionality
    %---------------------------------
    \begin{tikzpicture}
        \node [mybox] (box){%
            \begin{minipage}{0.247\textwidth}
                
            \end{minipage}
        };
        \node[fancytitle, right=10pt] at (box.north west) {Curse of Dimensionality};
    \end{tikzpicture}
    %---------------------------------
    
    %---------------------------------
    % Backpropagation
    %---------------------------------
    \begin{tikzpicture}
        \node [mybox] (box){%
            \begin{minipage}{0.247\textwidth}
                \textbf{MLP Layer}\\
                \textbf{Matrix Notation}\\
                \textbf{Full Batch}\\
                \textbf{Batch Normalization}\\
                \resizebox{\textwidth}{!}{
                    \begin{tabular}{|c|c|c|}
                        \hline
                        \textbf{BGD}                                           & \textbf{SGD}                                                              & \textbf{MBGD}                                                            \\
                        \hline Smooth                                          & Wiggling, needs smoothing                                                 & Slightly wiggling                                                        \\
                        \hline
                        Not wiggling                                           & Wiggles around minimum                                                    & Wiggles around minimum                                                   \\
                        \hline
                        Strictly decreasing cost                               & Not necessarily decreasing cost                                           & Typically decreasing cost                                                \\
                        \hline
                        Many epochs needed                                     & Few epochs needed                                                         & Less epochs than BGD, more than SGD needed                               \\
                        \hline
                        Choose larger learning rate                            & Choose smaller learning rate                                              & Choose medium learning rate (dependent on model)                         \\
                        \hline
                        No out-of-core support – all data in RAM (\(\sim m\)). & Out-of-core support - not all data to be kept in RAM of a single machine. & Out-of-core support - not all data to be kept in RAM of a single machine \\
                        \hline
                        Easy to parallelise                                    & Not easy to parallelise                                                   & Easy to parallelise                                                      \\
                        \hline
                    \end{tabular}
                }
            \end{minipage}
        };
        \node[fancytitle, right=10pt] at (box.north west) {Backpropagation};
    \end{tikzpicture}
    %---------------------------------
    
    %---------------------------------
    % Vanishing Exploding Gradient
    %---------------------------------
    \begin{tikzpicture}
        \node [mybox] (box){%
            \begin{minipage}{0.247\textwidth}
                \textbf{Saturation}\\
                \textbf{Variance Change}\\
                \textbf{Xavier \& Heu Initialization}\\
                \textbf{Batch Normalization}\\
                \textbf{Non Saturating Activation Function}\\
                \textbf{Gradient Clipping}\\
            \end{minipage}
        };
        \node[fancytitle, right=10pt] at (box.north west) {Vanishing Exploding Gradient};
    \end{tikzpicture}
    %---------------------------------
    
    %---------------------------------
    % Optimizers
    %---------------------------------
    \begin{tikzpicture}
        \node [mybox] (box){%
            \begin{minipage}{0.247\textwidth}
                \textbf{Momentum}\\
                \textbf{AdaGrad}\\
                \textbf{RMS Prop}\\
                \textbf{Adam}\\
                \textbf{Scheduler}\\
            \end{minipage}
        };
        \node[fancytitle, right=10pt] at (box.north west) {Optimizers};
    \end{tikzpicture}
    %---------------------------------
    
    %---------------------------------
    % Regularization
    %---------------------------------
    \begin{tikzpicture}
        \node [mybox] (box){%
            \begin{minipage}{0.247\textwidth}
                \textbf{Weight Penalty}\\
                \textbf{Dropout}\\
                \textbf{Early Stopping}\\
            \end{minipage}
        };
        \node[fancytitle, right=10pt] at (box.north west) {Regularization};
    \end{tikzpicture}
    %---------------------------------
    
    %---------------------------------
    % CNN
    %---------------------------------
    \begin{tikzpicture}
        \node [mybox] (box){%
            \begin{minipage}{0.247\textwidth}
                \textbf{Convolutional Layer}\\
                \textbf{Pooling Layer}\\
            \end{minipage}
        };
        \node[fancytitle, right=10pt] at (box.north west) {CNN};
    \end{tikzpicture}
    %---------------------------------
    
    %---------------------------------
    % Unbalanced Dataset
    %---------------------------------
    \begin{tikzpicture}
        \node [mybox] (box){%
            \begin{minipage}{0.247\textwidth}
                \textbf{Bayesian Approach}\\
                \textbf{Discrete}\\
                \textbf{Continuous}\\
                \textbf{Medical Test}\\
            \end{minipage}
        };
        \node[fancytitle, right=10pt] at (box.north west) {Unbalanced Dataset};
    \end{tikzpicture}
    %---------------------------------
    
    %---------------------------------
    % DeepCNN
    %---------------------------------
    \begin{tikzpicture}
        \node [mybox] (box){%
            \begin{minipage}{0.247\textwidth}
                \textbf{Conf2D Params}\\
                \textbf{MaxPooling}\\
                \textbf{LeNet5}\\
                \textbf{AlexNet}\\
                \textbf{VGGnet}\\
                \textbf{GoogleNet}\\
                \textbf{ResNet}\\
                \textbf{Pattern}\\
            \end{minipage}
        };
        \node[fancytitle, right=10pt] at (box.north west) {DeepCNN};
    \end{tikzpicture}
    %---------------------------------
    
    %---------------------------------
    % Feature Visualization
    %---------------------------------
    \begin{tikzpicture}
        \node [mybox] (box){%
            \begin{minipage}{0.247\textwidth}
                \textbf{Data Preparation}\\
                \textbf{Network}\\
                \textbf{Compile}\\
                \textbf{Evaluate}\\
                \textbf{Activation Map}\\
            \end{minipage}
        };
        \node[fancytitle, right=10pt] at (box.north west) {Feature Visualization};
    \end{tikzpicture}
    %---------------------------------
    
    %---------------------------------
    % Data Augmentation
    %---------------------------------
    \begin{tikzpicture}
        \node [mybox] (box){%
            \begin{minipage}{0.247\textwidth}
                \textbf{Principle}\\
                \textbf{Types}\\
                \textbf{Strategies}\\
                \textbf{Keras}\\
            \end{minipage}
        };
        \node[fancytitle, right=10pt] at (box.north west) {Data Augmentation};
    \end{tikzpicture}
    %---------------------------------
    
    %---------------------------------
    % Functional API
    %---------------------------------
    \begin{tikzpicture}
        \node [mybox] (box){%
            \begin{minipage}{0.247\textwidth}
                \textbf{Sequential vs Functionals}\\
                \textbf{Architecture 1}\\
                \textbf{Architecture 2}\\
                \textbf{Architecture 3}\\
            \end{minipage}
        };
        \node[fancytitle, right=10pt] at (box.north west) {Functional API};
    \end{tikzpicture}
    %---------------------------------
    
    %---------------------------------
    % Transfer Learning
    %---------------------------------
    \begin{tikzpicture}
        \node [mybox] (box){%
            \begin{minipage}{0.247\textwidth}
                \textbf{Principle}\\
                \textbf{Keras Code}\\
                \textbf{MobileNet}\\
                \textbf{Strategies}\\
            \end{minipage}
        };
        \node[fancytitle, right=10pt] at (box.north west) {Transfer Learning};
    \end{tikzpicture}
    %---------------------------------
    
    %---------------------------------
    % RNN
    %---------------------------------
    \begin{tikzpicture}
        \node [mybox] (box){%
            \begin{minipage}{0.247\textwidth}
                \textbf{Use Case}\\
                \textbf{Model Category}\\
                \textbf{Recurrence Net}\\
                \textbf{Single Layer}\\
                \textbf{Many to Many}\\
                \textbf{Un exemple par catégorie}\\
                \textbf{Stacked RNN}\\
            \end{minipage}
        };
        \node[fancytitle, right=10pt] at (box.north west) {RNN};
    \end{tikzpicture}
    %---------------------------------
    
    %---------------------------------
    % LSTM
    %---------------------------------
    \begin{tikzpicture}
        \node [mybox] (box){%
            \begin{minipage}{0.247\textwidth}
                \textbf{Long Term Memory Unit Cell}\\
                \textbf{Gates}\\
                \textbf{Backprop}\\
                \textbf{Keras}\\
                \textbf{GRE}\\
            \end{minipage}
        };
        \node[fancytitle, right=10pt] at (box.north west) {LSTM};
    \end{tikzpicture}
    %---------------------------------
    
    %---------------------------------
    % Word Embedding
    %---------------------------------
    \begin{tikzpicture}
        \node [mybox] (box){%
            \begin{minipage}{0.247\textwidth}
                \textbf{Word}\\
                \textbf{Training}\\
            \end{minipage}
        };
        \node[fancytitle, right=10pt] at (box.north west) {Word Embedding};
    \end{tikzpicture}
    %---------------------------------
    
    %---------------------------------
    % Sentiment Classification
    %---------------------------------
    \begin{tikzpicture}
        \node [mybox] (box){%
            \begin{minipage}{0.247\textwidth}
                \textbf{Strategy}\\
                \textbf{Architecture}\\
            \end{minipage}
        };
        \node[fancytitle, right=10pt] at (box.north west) {Sentiment Classification};
    \end{tikzpicture}
    %---------------------------------
    
    %---------------------------------
    % Autoencoder
    %---------------------------------
    \begin{tikzpicture}
        \node [mybox] (box){%
            \begin{minipage}{0.247\textwidth}
                \textbf{Definition}\\
                \textbf{Use Case}\\
            \end{minipage}
        };
        \node[fancytitle, right=10pt] at (box.north west) {Autoencoder};
    \end{tikzpicture}
    %---------------------------------
    
    %---------------------------------
    % GenRNN
    %---------------------------------
    \begin{tikzpicture}
        \node [mybox] (box){%
            \begin{minipage}{0.247\textwidth}
                \textbf{Many to Many}\\
                \textbf{Many to One}\\
            \end{minipage}
        };
        \node[fancytitle, right=10pt] at (box.north west) {GenRNN};
    \end{tikzpicture}
    %---------------------------------
    
    %---------------------------------
    % Attention
    %---------------------------------
    \begin{tikzpicture}
        \node [mybox] (box){%
            \begin{minipage}{0.247\textwidth}
                \textbf{Sequence to Sequence}\\
                \textbf{Attention}\\
            \end{minipage}
        };
        \node[fancytitle, right=10pt] at (box.north west) {Attention};
    \end{tikzpicture}
    %---------------------------------
    
    %---------------------------------
    % Transformer
    %---------------------------------
    \begin{tikzpicture}
        \node [mybox] (box){%
            \begin{minipage}{0.247\textwidth}
                \textbf{High-Level Architecture}\\
                \textbf{Self-Attention}\\
                \textbf{Full Architecture}\\
            \end{minipage}
        };
        \node[fancytitle, right=10pt] at (box.north west) {Transformer};
    \end{tikzpicture}
    %---------------------------------
    
\end{multicols*}

\end{document}

