%
% Packages
%
\documentclass{article}
\usepackage[landscape]{geometry}
\usepackage{url}
\usepackage{multicol}
\usepackage{amsmath}
\usepackage{esint}
\usepackage{bigints}
\usepackage{amsfonts}
\usepackage{xcolor}
\usepackage{tikz}
\usetikzlibrary{calc}
\usetikzlibrary{decorations.pathmorphing}
\usepackage{amsmath,amssymb}
\usepackage{colortbl}
\usepackage{xcolor}
\usepackage{mathtools}
\usepackage{amsmath,amssymb}
\usepackage{enumitem}
\usepackage{xhfill}
\usepackage[french]{babel}
\usepackage[utf8]{inputenc}
\usepackage{parskip}
\usepackage[T1]{fontenc}
\usepackage{mathrsfs}
\makeatletter

%
% Math
%
\newcommand{\Real}{\mathbb R}
\newcommand{\RPlus}{\Real^{+}}
\newcommand{\norm}[1]{\left\Vert#1\right\Vert}
\newcommand{\abs}[1]{\left\vert#1\right\vert}
\newcommand{\setn}[1]{\left\{#1\right\}_{\scriptscriptstyle n \ge 1}}
\newcommand{\set}[1]{\left\{#1\right\}}
\newcommand{\seq}[1]{\left<#1\right>}
\newcommand{\eps}{\varepsilon}
\newcommand{\To}{\longrightarrow}
\newcommand{\Prob}{\rm{P}}
\newcommand{\F}{\mathcal{F}}
\newcommand{\h}{\mathcal{H}}
\newcommand{\M}{\mathcal{M}}
\newcommand{\X}{\mathcal{X}}
\newcommand{\N}{\mathcal{N}}
\newcommand{\E}{{\rm E}}
\newcommand{\Hnull}{{\rm H}_{0}}
\newcommand{\Hone}{{\rm H}_{1}}
\newcommand{\Var}{{\rm Var}}
\newcommand{\Cov}{{\rm Cov}}
\newcommand{\sign}{{\rm sign}}
\newcommand{\med}{{\rm med}}
\newcommand{\tr}{{\rm tr}}
\newcommand{\T}{{\text{\tiny \rm T}}}
\newcommand{\minf}{- \, \infty}
\newcommand{\intervalle}[4]{\mathopen{#1}#2\mathpunct{},#3\mathclose{#4}}
\newcommand{\intervalleff}[2]{\intervalle{[}{#1}{#2}{]}}
\newcommand{\intervalleof}[2]{\intervalle{]}{#1}{#2}{]}}
\newcommand{\intervallefo}[2]{\intervalle{[}{#1}{#2}{[}}
\newcommand{\intervalleoo}[2]{\intervalle{]}{#1}{#2}{[}}
\newcommand*\conj[1]{\overline{#1}}
\newcommand*\mean[1]{\overline{#1}}

%
% Setup
%
\title{MachLe Résumé Olivier D'Ancona}
\advance\topmargin-.8in
\advance\textheight3in
\advance\textwidth3in
\advance\oddsidemargin-1.5in
\advance\evensidemargin-1.5in
\parindent0pt
\parskip2pt

%
% Commands
%
\newcommand*\bigcdot{\mathpalette\bigcdot@{.5}}
\newcommand*\bigcdot@[2]{\mathbin{\vcenter{\hbox{\scalebox{#2}{$\m@th#1\bullet$}}}}}
\makeatother
\newcommand{\hr}{\centerline{\rule{3.5in}{1pt}}}
%\colorbox[HTML]{e4e4e4}{\makebox[\textwidth-2\fboxsep][l]{texto}
\newcommand{\nc}[2][]{%
\vspace{-.1cm}
\tikz \draw [draw=black, ultra thick, #1]
    ($(current page.center)-(0.5\linewidth,0)$) --
    ($(current page.center)+(0.5\linewidth,0)$)
    node [midway, fill=white] {#2};
}% tomado de https://tex.stackexchange.com/questions/179425/a-new-command-of-the-form-tex


%
% Styles
%
\tikzstyle{mybox} = [draw=black, fill=white, very thick,
rectangle, rounded corners, inner sep=2pt, inner ysep=7pt]
\tikzstyle{fancytitle} =[fill=black, text=white, font=\bfseries]

\newlength{\boxsize}
\setlength{\boxsize}{0.24\textwidth}

%###############################################################################################
%
%                                         Document
%
%###############################################################################################

\begin{document}

%---------------------------------
% Title
%---------------------------------
\begin{center}
    {\huge{\textbf{MachLe - Résumé Olivier D'Ancona}}}\\
\end{center}

\begin{multicols*}{4}

    %---------------------------------
    % Evaluation Metrics
    %---------------------------------
    \begin{tikzpicture}
        \node [mybox] (box){%
            \begin{minipage}{0.31\textwidth}
            $Accuracy = \dfrac{TP + TN}{ TP + TN + FP + FN}$\\
            $Precision = \dfrac{TP}{ TP + FP}$\\
            $Recall = \dfrac{TP}{ TP + FN}$\\
            $Specificity = \dfrac{TP}{TN + FP}$\\
            $Fscore = \dfrac{2\cdot Precision \cdot Recall}{ Precision + Recall}$\\
           \end{minipage}
        };
        \node[fancytitle, right=10pt] at (box.north west) {Evaluation Metrics};
    \end{tikzpicture}
    %---------------------------------


    %---------------------------------
    % Activation Functions
    %---------------------------------
    \begin{tikzpicture}
        \node [mybox] (box){%
            \begin{minipage}{0.31\textwidth}
                \textbf{Sigmoid}: $\dfrac{1}{1 + e^{-x}}$\\
                \textbf{Hyperbolic tangent}: $\frac {e^{x} - e^{-x}}{e^{x} + e^{-x}}$\\
                \textbf{Relu}: $\begin{cases}0 & \text{si $x < 0$ } \\x & \text{si $x \ge 0$} \\\end{cases}$\\
                \textbf{Gaussian}: $e^{-x^{2}}$\\
                \textbf{Softmax}: $\dfrac{e^{z_j}}{\displaystyle \sum_{k=1}^{K}{e^{z_k}}}$\\
            \end{minipage}
        };
        \node[fancytitle, right=10pt] at (box.north west) {Activation Functions};
    \end{tikzpicture}
    %---------------------------------


    %---------------------------------
    % Neural Network
    %---------------------------------
    \begin{tikzpicture}
        \node [mybox] (box){%
            \begin{minipage}{0.31\textwidth}
                \nc{Structure}\\
                \textbf{Biais}: $b$, An extra weight that can be learned using a learning algorithm. The purpose is to replace threshold.\\
                \textbf{Input}: $I$, Input vector
                \textbf{Weights}: $W$, Vector of weights
                \nc{Learning algorithm}\\
                \vspace{-0.8cm}
                \begin{enumerate}
                \item Randomly initialize weights
                \item Compute the neuron's output for a fiven input vector X
                \item Update weights: $W_j(t+1) = W_j(t) + \eta\left(\hat{y_i}-y\right)x$ with $\eta$ the learning rate and $\hat{y_i}$ the desired output.
                \item Repeat steps 2 and 3 for the number of epochs you need or until the error is smaller than a threshold.
                \end{enumerate}
           \end{minipage}
        };
        \node[fancytitle, right=10pt] at (box.north west) {Neural Network};
    \end{tikzpicture}
    %---------------------------------


    %---------------------------------
    % KNN for classification and regression
    %---------------------------------
    \begin{tikzpicture}
        \node [mybox] (box){%
            \begin{minipage}{0.31\textwidth}
            \end{minipage}
        };
        \node[fancytitle, right=10pt] at (box.north west) {KNN for classification and regression};
    \end{tikzpicture}
    %---------------------------------

    %---------------------------------
    % Bayesian approaches / GMMs / System evaluation
    %---------------------------------
    \begin{tikzpicture}
        \node [mybox] (box){%
            \begin{minipage}{0.31\textwidth}
            \end{minipage}
        };
        \node[fancytitle, right=10pt] at (box.north west) {Bayesian approaches / GMMs / System evaluation};
    \end{tikzpicture}
    %---------------------------------


    %---------------------------------
    % Linear Regression
    %---------------------------------
    \begin{tikzpicture}
        \node [mybox] (box){%
            \begin{minipage}{\boxsize}
                Soit un tableau de données: 

                $x$ = Soap(g) , $y$ = Height(cm) , $x \cdot y$ , $x^2$	  

                $$ X = [1, Soap] $$
                $$ X^TX = \begin{bmatrix} n & \sum{x_i} \\ \sum{x_i} & \sum{x_i^2} \end{bmatrix} = \begin{bmatrix} 7 & 38.5 \\ 38.5 & 218.95 \end{bmatrix} $$
                $$ X^Ty = \begin{bmatrix} \sum{y_i} \\ \sum{x_iy_i} \end{bmatrix} = \begin{bmatrix} 348 \\ 1975 \end{bmatrix} $$
                $$ \hat{\theta} = (X^TX)^{-1}X^Ty = \begin{bmatrix} -2.67 \\ 9.51 \end{bmatrix} $$
                Inverse d'une matrice 2x2 :

                $$ \begin{bmatrix} a & b \\ c & d \end{bmatrix}^{-1} = \dfrac{1}{ad-bc} \begin{bmatrix} d & -b \\ -c & a \end{bmatrix} $$
            \end{minipage}
        };
        \node[fancytitle, right=10pt] at (box.north west) {Linear Regression};
    \end{tikzpicture}
    %---------------------------------

    %---------------------------------
    % System engineering / Data preparation / Tuning
    %---------------------------------
    \begin{tikzpicture}
        \node [mybox] (box){%
            \begin{minipage}{0.31\textwidth}
            \end{minipage}
        };
        \node[fancytitle, right=10pt] at (box.north west) {System engineering / Data preparation / Tuning};
    \end{tikzpicture}
    %---------------------------------

    %---------------------------------
    % Logistic regression / Overfitting
    %---------------------------------
    \begin{tikzpicture}
        \node [mybox] (box){%
            \begin{minipage}{0.31\textwidth}
            \end{minipage}
        };
        \node[fancytitle, right=10pt] at (box.north west) {Logistic regression / Overfitting};
    \end{tikzpicture}
    %---------------------------------


    %---------------------------------
    % SVM
    %---------------------------------
    \begin{tikzpicture}
        \node [mybox] (box){%
            \begin{minipage}{0.31\textwidth}
            \end{minipage}
        };
        \node[fancytitle, right=10pt] at (box.north west) {SVM};
    \end{tikzpicture}
    %---------------------------------

    %---------------------------------
    % Clustering
    %---------------------------------
    \begin{tikzpicture}
        \node [mybox] (box){%
            \begin{minipage}{0.31\textwidth}
            \end{minipage}
        };
        \node[fancytitle, right=10pt] at (box.north west) {Clustering};
    \end{tikzpicture}
    %---------------------------------


    %---------------------------------
    % Decision Trees
    %---------------------------------
    \begin{tikzpicture}
        \node [mybox] (box){%
            \begin{minipage}{0.31\textwidth}
            \end{minipage}
        };
        \node[fancytitle, right=10pt] at (box.north west) {Decision Trees};
    \end{tikzpicture}
    %---------------------------------

    %---------------------------------
    % Convolutional Neural Networks
    %---------------------------------
    \begin{tikzpicture}
        \node [mybox] (box){%
            \begin{minipage}{0.31\textwidth}
            \end{minipage}
        };
        \node[fancytitle, right=10pt] at (box.north west) {Convolutional Neural Networks};
    \end{tikzpicture}
    %---------------------------------

    %---------------------------------
    % Recurrent Neural Networks
    %---------------------------------
    \begin{tikzpicture}
        \node [mybox] (box){%
            \begin{minipage}{0.31\textwidth}
            \end{minipage}
        };
        \node[fancytitle, right=10pt] at (box.north west) {Recurrent Neural Networks};
    \end{tikzpicture}
    %---------------------------------

    %---------------------------------
    % Dimensionality Reduction
    %---------------------------------
    \begin{tikzpicture}
        \node [mybox] (box){%
            \begin{minipage}{0.31\textwidth}
            \end{minipage}
        };
        \node[fancytitle, right=10pt] at (box.north west) {Dimensionality Reduction};
    \end{tikzpicture}
    %---------------------------------

    %---------------------------------
    % Reinforcement Learning
    %---------------------------------
    \begin{tikzpicture}
        \node [mybox] (box){%
            \begin{minipage}{0.31\textwidth}
            \end{minipage}
        };
        \node[fancytitle, right=10pt] at (box.north west) {Reinforcement Learning};
    \end{tikzpicture}
    %---------------------------------










    
%###############################################################################################
%                                         Deuxième Page
%###############################################################################################


\end{multicols*}
    
\end{document}
